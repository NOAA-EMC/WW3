\subsubsection{Data in {\F w3adatmd}.} \label{sub:adatmd}

The module {\F w3adatmd} in the file {\file w3adatmd.ftn} contains data that
are used inside the wave model only. Embedded in the module are the following
parameters:

\begin{vlist}
\vit{nadata}{i }{Number of models in array dim.}
\vit{iadata}{i }{Selected model for output, init. at -1.}
\vit{mpibuf}{IP}{Number of buffer arrays for 'hidden' MPI communications
                (no hiding for {\F mpibuf = 1}).}
\vit{wadat }{t }{Basic data structure.}
\vit{wadats}{ta}{Array of data structures.}
\end{vlist}

\noindent
For each element of the type {\F wadat}, alias pointers are defined as in the
module {\F w3adatmd}. Within the type, several groups of variables are
present. The first are the internal grid definition of the model:

\begin{vlist}
\vit{cg    }{ra}{Group velocities for all wave model sea points and 
                 frequencies.}
\vit{wn    }{ra}{Idem, wavenumbers.}
\end{vlist}

\noindent
The second group of parameters consists auxiliary arrays needed to process
model input:

\begin{vlist}
\vit{ca0-i  }{ra}{Absolute current velocity (initial and inc.) in \\
                  {\F w3ucur}. \hfill (m/s)}
\vit{cd0-i  }{ra}{Current direction (initial and increment) in \\
                  {\F w3ucur}. \hfill (rad)}
\vit{ua0-i  }{ra}{Absolute wind speeds (init. and incr.) in 
                  {\F w3uwnd} \hfill (m/s)}
\vit{ud0-i  }{ra}{Wind direction (initial and incr.) in
                  {\F w3uwnd} \hfill (rad)}
\vit{as0-i  }{ra}{Stability par. (initial and incr.) in
                  {\F w3uwnd} \hfill (degr)}
\vit{atrnx-y}{ra}{Actual transparency info.}
\end{vlist}

\noindent
The third group of parameters consists of internal gridded fields of
parameters:

\begin{vlist}
\vit{dw   }{ra}{Water depths.}
\vit{ua   }{ra}{Absolute wind speeds.}
\vit{ud   }{ra}{Absolute wind direction.}
\vit{u10  }{ra}{Wind speed used.}
\vit{u10d }{ra}{Wind direction used.}
\vit{as   }{ra}{Stability parameter.}
\vit{cx/y }{ra}{Current components.}
\vit{emn  }{ra}{Mean energy.}
\vit{fmn  }{ra}{Mean frequency.}
\vit{wnm  }{ra}{Mean wavenumber.}
\vit{amx  }{ra}{Spectral maximum.}
\vit{cds  }{ra}{Drag coefficient.}
\vit{z0s  }{ra}{Roughness parameter.}
\vit{hs   }{ra}{Wave height.}
\vit{wlm  }{ra}{Mean wave length.}
\vit{tmn  }{ra}{Mean wave period.}
\vit{thm  }{ra}{Mean wave direction.}
\vit{ths  }{ra}{Mean directional spread.}
\vit{fp0  }{ra}{Peak frequency.}
\vit{thp0 }{ra}{Peak direction.}
\vit{fp1  }{ra}{Wind sea peak frequency.}
\vit{thp1 }{ra}{Wind sea peak direction.}
\vit{dtdyn}{ra}{Mean dynamic time step (raw).}
\vit{fcut }{ra}{Cut-off frequency for tail.}
\vit{aba  }{ra}{Near bottom rms wave excursion amplitude.}
\vit{abd  }{ra}{Corresponding direction.}
\vit{uba  }{ra}{Near bottom rms wave velocity.}
\vit{ubd  }{ra}{Corresponding direction.}
\vit{s{\it xx}}{ra}{Radiation stress components.}
\vit{phs  }{ra}{Wave height of partition of spectrum.}
\vit{ptp  }{ra}{Peak period of partition of spectrum.}
\vit{plp  }{ra}{Peak wave length of partition of spectrum.}
\vit{pth  }{ra}{Direction of partition of spectrum.}
\vit{psi  }{ra}{Directional spread of partition of spectrum.}
\vit{pws  }{ra}{Wind sea fraction of partition of spectrum.}
\vit{pwst }{ra}{Wind sea fraction of total spectrum.}
\vit{pnr  }{ra}{Number of wave fields in partitioning.}
\vit{ddd{\it x} }{ra}{Spatial derivatives of the depth.}
\vit{dc{\it x}d{\it x}}{ra}{Spatial derivatives of the current.}
%\vit{drat }{ra}{Ratio or air and water densities.}
%\vit{tauo/y}{ra}{Momentum flux from waves to ocean.}
%\vit{tauw/y}{ra}{Momentum flux from atmosphere to waves.}
%\vit{phat}{ra}{Wave-modified mean pressure (responsible for set-up).}
%\vit{z0w}{ra}{Under-water roughness length.}
%\vit{phiaw}{ra}{Atmosphere to waves energy flux.}
%\vit{phioc}{ra}{Waves to ocean energy flux.}
%\vit{tusx/y}{ra}{Depth-integrated wave Stokes transport.}
%\vit{ussx/y}{ra}{Surface Stokes drift.}
%\vit{mssx/y}{ra}{Mean square slope components.}
%\vit{usero}{ra}{Slots for user-supplied output parameters.}
\end{vlist}

\noindent
The fourth group of parameters consists of map data for the first
order propagation scheme ({\F !/pr1}).

\begin{vlist}
\vit{is0-2 }{ia}{Spectral propagation maps.}
\vit{facvx-y}{ra}{Spatial propagation factor map.}
\end{vlist}

\noindent
The fifth group of parameters consists of map data for the third
order propagation scheme ({\F !/pr2-4}).

\begin{vlist}
\vit{nmx-y{\it{n}}}{i }{Counters for {\F mapx2}, see {\F w3map3}.}
\vit{nmxy  }{i }{Dimension of {\F mapxy}.}
\vit{nact1-2}{i }{Dimension of {\F mapaxy}.}
\vit{ncent }{i }{Dimension of {\F mapaxy}.}
\vit{mapx2 }{ia}{Map for prop. in 'x' (longitude) dir.}
\vit{mapy2 }{ia}{Idem in y' (latitude) direction.}
\vit{mapxy }{ia}{}
\vit{mapaxy}{ia}{List of active points used in {\F w3qck1}.}
\vit{mapcxy}{ia}{List of central points used in avg.}
\vit{mapth2}{ia}{Like {\F mapx2} for refraction (rotated and
                 shifted, see {\F w3ktp3}).
                 Like {\F mapaxy}.}
\vit{mapwn2}{ia}{Like {\F mapx2} for wavenumber shift.}
\vit{maptrn}{la}{Map to block out GSE mitigation in proper grid
                 points.}
\end{vlist}

\noindent 
The sixth group of parameters consists variables used by the parameterizations
for the nonlinear interactions :

\begin{vlist}
\vit{nfr        }{i }{Number of frequencies (\F{ nfr = nk } ).
                                                       \hfill ({\F !/nl1})}
\vit{nfrhgh     }{i }{Auxiliary frequency counter.     \hfill ({\F !/nl1})}
\vit{nfrchg     }{i }{Id.                              \hfill ({\F !/nl1})}
\vit{nspecx-y   }{i }{Auxiliary spectral counter.      \hfill ({\F !/nl1})}
\vit{ip\it{nn}  }{ia}{Spectral address for $S_{nl}$.   \hfill ({\F !/nl1})}
\vit{im\it{nn}  }{ia}{Id.                              \hfill ({\F !/nl1})}
\vit{ic\it{nn}  }{ia}{Id.                              \hfill ({\F !/nl1})}
\vit{dal\it{n}  }{r }{Lambda dependent weight factors. \hfill ({\F !/nl1})}
\vit{awg\it{n}  }{r }{Interpolation weights for Snl.   \hfill ({\F !/nl1})}
\vit{swg\it{n}  }{r }{Interpolation weights for diag. term.
                                                       \hfill ({\F !/nl1})}
\vit{af11       }{ra}{Scaling array ($f^{11}$).        \hfill ({\F !/nl1})}
\vit{nlinit     }{l }{Flag for initialization.         \hfill ({\F !/nl1})}
\end{vlist}

\noindent
The seventh group of parameters consists MPP and MPI variables:

\begin{vlist}
\vit{iappro}{ia}{Processor numbers for propagation calc. for each
                 spectral component.}
\vit{mpi\_comm\_wave, mpi\_comm\_wcmp}{}{}
\vit{      }{i }{mpi communicator for wave model.     \hfill ({\F !/mpi})}
\vit{ww3\_field\_vec, ww3\_spec\_vec}{}{}
\vit{      }{i }{MPI derived vector types.            \hfill ({\F !/mpi})}
\vit{nrqsg1}{i }{Number of handles in {\F irqsg1}.    \hfill ({\F !/mpi})}
\vit{nrqsg2}{i }{Number of handles in {\F irqsg2}.    \hfill ({\F !/mpi})}
\vit{ibfloc}{i }{Present active buffer number.        \hfill ({\F !/mpi})}
\vit{isploc}{i }{Corresponding local spectral bin number \\
                 {\F (1,nsploc,1)}.                   \hfill ({\F !/mpi})}
\vit{nsploc}{i }{Total number of spectral bins for which prop. \\ is
                 performed on present CPU.            \hfill ({\F !/mpi})}
\vit{bstat }{ia}{Status of buffer (size {\F mpibuf}): \hfill ({\F !/mpi})}
\vit{      }{  }{$\:\:$ 0: Inactive.}
\vit{      }{  }{$\:\:$ 1: {\F a} $\to$ {\F store} (active or finished).}
\vit{      }{  }{$\:\:$ 2: {\F store} $\to$ {\F a} (active or finished).}
\vit{bispl }{ia}{Local spectral bin number for buffer
                 (size {\F mpibuf}).                  \hfill ({\F !/mpi})}
\vit{irqsg1}{ia}{MPI request handles for scatters and gathers \\ to {\F va}
                (persistent).                         \hfill ({\F !/mpi})}
\vit{irqsg2}{ia}{MPI request handles for gathers and scatters \\ to
                 {\F STORE} (persistent).             \hfill ({\F !/mpi})}
\vit{gstore, sstore}{}{}
\vit{      }{ra}{Communication buffer {\F (nsea,mpibuf)}.
                                                      \hfill ({\F !/mpi})}
\vit{sppnt }{ra}{Communication buffer {\F (nth,nk,4)}.}
\end{vlist}

\noindent
The final group of parameters consist of all other internal auxiliary
parameters that need to be saved between model runs:

\begin{vlist}
\vit{itime }{i }{Discrete time step counter.}
\vit{ipass }{i }{Pass counter for log file.}
\vit{idlast}{i }{Last day ID for log file.}
\vit{nsealm}{i }{Maximum number of local sea points.}
\vit{alpha }{ra}{Phillips' alpha.}
\vit{flcold}{l }{Flag for 'cold start' of model.}
\vit{fliwnd}{l }{Flag for initialization of model based on wind.}
\vit{ainit }{l }{Flag for array initialization.}
\vit{fl\_all}{l}{Flag for all or partial array initialization.}
\end{vlist}
