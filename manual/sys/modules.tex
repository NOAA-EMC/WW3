\vsssub
\subsubsection{~Parameter settings in modules}
\vsssub

Several modules has internally used parameter settings. Here only parameter
settings that are generally usable or impact model output are presented.

\vspace{\baselineskip} \noindent
Physical and mathematical constants : \hfill {\file constants.ftn}
\begin{vlist}
\vit{grav  }{rp}{Acceleration of gravity $g$.
                \hfill (m s$^{-2}$)}
\vit{dwat  }{rp}{Density of water. \hfill(kg m$^{-3}$)}
\vit{dair  }{rp}{Density of air. \hfill(kg m$^{-3}$)}
\vit{pi    }{rp}{$\pi$.}
\vit{tpi   }{rp}{$2\pi$.}
\vit{hpi   }{rp}{$0.5\pi$.}
\vit{tpiinv}{rp}{$(2\pi)^{-1}$.}
\vit{hpiinv}{rp}{$(0.5\pi)^{-1}$.}
\vit{rade  }{rp}{Conversion factor from radians to degrees.}
\vit{dera  }{rp}{Conversion factor from degrees to radians.}
\vit{radius}{rp}{Radius of the earth. \hfill (m)}
\vit{g2pi3i}{rp}{$g^{-2} (2\pi)^{-3}$.}
\vit{g1pi1i}{rp}{$g^{-1}(2\pi)^{-1}$.}
\end{vlist}

\noindent
Wave model initialization module : \hfill {\file w3initmd.ftn}
\begin{vlist}
\vit{critos}{rp}{Critical fraction of resources used for output only
                     (triggers warning output).}
\vit{wwver }{cp}{Version number of the main program.}
\end{vlist}

\noindent
I/O module ({\file mod\_def.ww3}) : \hfill {\file w3iogrmd.ftn}
\begin{vlist}
\vit{vergrd}{cp\opt}{Version number of file {\file mod\_def.ww3}.}
\vit{idstr }{cp\opt}{ID string for file.}
\end{vlist}

\noindent
I/O module ({\file out\_grd.ww3}) : \hfill {\file w3iogomd.ftn}
\begin{vlist}
\vit{verogr}{cp\opt}{Version number of file {\file out\_grd.ww3}.}
\vit{idstr }{cp\opt}{ID string for file.}
\end{vlist}

\noindent
I/O module ({\file out\_pnt.ww3}) : \hfill {\file w3iopomd.ftn}
\begin{vlist}
\vit{veropt}{cp\opt}{Version number of file {\file out\_pnt.ww3}.}
\vit{idstr }{cp\opt}{ID string for file.}
\end{vlist}

\noindent
I/O module ({\file track\_o.ww3}) : \hfill {\file w3iotrmd.ftn}
\begin{vlist}
\vit{vertrk}{cp\opt}{Version number of file {\file track\_o.ww3}.}
\vit{idstri}{cp\opt}{ID string for file {\file track\_i.ww3}.}
\end{vlist}

\noindent
I/O module ({\file restart.ww3}) : \hfill {\file w3iorsmd.ftn}
\begin{vlist}
\vit{verini}{cp\opt}{Version number of file {\file restart.ww3}.}
\vit{idstr }{cp\opt}{ID string for file.}
\end{vlist}

\noindent
I/O module ({\file nest.ww3}) : \hfill {\file w3iobcmd.ftn}
\begin{vlist}
\vit{verbpt}{cp\opt}{Version number of file {\file nest.ww3}.}
\vit{idstr }{cp\opt}{ID string for file.}
\end{vlist}

\noindent
I/O module ({\file partition.ww3}) : \hfill {\file w3iosfmd.ftn}
\begin{vlist}
\vit{vertrt}{cp\opt}{Version number of file {\file partition.ww3}.}
\vit{idstr }{cp\opt}{ID string for file.}
\end{vlist}

\noindent
Multi-grid model input update : \hfill {\file wmupdtmd.ftn}
\begin{vlist}
\vit{swpmax}{ip}{Maximum number of extrapolation sweeps allowed to make maps
                 match in conversion from input from input grid to wave model
                 grid.}
\end{vlist}

\vspace{\baselineskip} \noindent
The service modules contain private variables only, with the exception of the
interpolation tables for the solution of the dispersion relation

\vspace{\baselineskip} \noindent
Solving the dispersion relation : \hfill {\file w3dispmd.ftn}
\begin{vlist}
\vit{nar1d }{ip}{Dimension of interpolation tables.}
\vit{dfac  }{rp}{Maximum nondimensional water depth $kd$.}
\vit{ecg1  }{ra}{Table for calculating  group velocities from
                 the frequency and the depth.}
\vit{ewn1  }{ra}{Id. wavenumbers.}
\vit{n1max }{i }{Largest index in tables.}
\vit{dsie  }{r }{Nondimensional frequency increment.}
\end{vlist}

\noindent
Automatic unit number assignment : \hfill {\file wmunitmd.ftn}
\begin{vlist}
\vit{unitlw}{ip}{Lowest unit number to be considered.}
\vit{unithg}{ip}{Highest unit number to be considered.}
\vit{inplow, inphgh}{}{}
\vit{      }{ip}{Range of input file unit numbers.}
\vit{outlow, outhgh}{}{}
\vit{      }{ip}{Range of output file unit numbers.}
\vit{scrlow, scrhgh}{}{}
\vit{      }{ip}{Range of scratch file unit numbers.}
\end{vlist}

\noindent
Note that the main programs only contain locally defined variables, which need
not be documented here in detail.

