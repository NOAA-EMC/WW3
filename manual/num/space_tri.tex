\vssub
\subsubsection{~Triangular unstructured grids} \label{sub:num_space_tri}
\conthead{WWM-II}{A. Roland, F. Ardhuin, M. Dutour-Sikiri\'c}

\noindent 
Triangle-based grids can be used in \ws\ by using numerical schemes based on
contour residual distribution (RD) \citep[see][for a review]{rep:Roland2008}.
These efficient schemes have initially been implemented in the Wind Wave Model-II (WWM-II) and 
have subsequently been evaluated in WWIII \citep[e.g.][]{art:Aea09,
art:Mea10}.  This option is activated by setting the grid string to `{\code
UNST}' in {\file ww3\_grid.inp}.  Four schemes have been implemented, and the
choice of one or the other is done with the {\code UNST} namelist.  These are
the CRD-N-scheme (1st order), the CRD-PSI-scheme (better than 1st order, 2nd
order on triangular structured grids), the CRD-FCT-scheme (2nd order
space-time), and the implicit N-scheme. The default is the most efficient but
diffusive explicit N-scheme. An implicit variant of the RD-Schemes
using the method of lines and the N-Scheme for the space discretization was
implemented in the SWAN model by \cite{art:Zij10}. We note that these advection 
schemes do not include corrections for the garden sprinkler effect (GSE). These 
can be particularly visible for waves going around islands surrounded by deep 
water. In that case, the diffusion of the N scheme can compensate the GSE.

In practice the grid can be easily generated, using the PolyMesh interface
(software developed by Aron Roland), from a shoreline polygons database
\citep[e.g.][]{art:WS96} and a list of depth soundings, regular or irregular.

In this method the evolution of the spectrum at the nodes, where it is
evaluated, is based on the redistribution over the nodes of the flux
convergence into the median dual cells associated with the nodes (see Figure
\ref{fig:triangles}).  For any spectral component, the advection equation, Eq.
(\ref{eq:step_xy_prop}), is solved on the median dual cells: the incoming flux
into a cell gives the rate of change of the wave action at the corresponding
node. The various schemes implemented have different discretizations for the
estimation of this flux. The schemes have been presented in \citep[see][for a
review]{rep:Roland2008} and \citet{pro:Rol12}.

The equivalent of the CFL condition for explicit finite difference schemes 
on regular grids is the ratio of the
dual cell area divided by the product of the time step and all positive flux
into the dual cell. Because the spectral levels are imposed on the boundary
for the positive fluxes, the boundary nodes are excluded from this CFL
calculation and the incoming energy is set to zero, whereas the outgoing energy 
is fully absorbed. 

The boundary condition at the shoreline depends on the wave direction
relative to the shoreline orientation. This particular treatment is enforced
using the `{\code IOBPD}' array which is updated whenever the grid points
status map `{\code MAPSTA}' changes. The grid geometry is also used to define
local gradients of the water depth and currents. All other operations, such as
interpolation of the forcing on the grid and interpolation from the grid onto
output locations, is performed using linear interpolation in triangles.

All the triangle geometry operations assume a locally flat Earth. Depth and
current gradients on the grid are estimated at the nodes by weighting with
their angle the gradients over each triangle connected to the node.

\begin{figure} \begin{center}
\epsfig{file=./num/grid_triangles.eps,angle=0,width=4.in}
\caption{Example of a region of a triangle-based mesh, with in this case the
 small Island of Bannec, France. If the depth is greater than the minimum
depth, the nodes of the shoreline are active. These are characterized by a
larger number of neighbor nodes (6 in the example chosen) than neighbor
triangles (5 in the same example).}
\label{fig:triangles} \botline
\end{center}
\end{figure}

