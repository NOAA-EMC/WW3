\addcontentsline{toc}{subsubsection}{\strut \hspace{24mm} First-order scheme}

\vspace{\baselineskip} 
\vspace{\baselineskip} 
\noindent {\bf First-order scheme}

\opthead{PR1}{\ws}{H. L. Tolman}

\noindent
A simple and cheap first order upwind scheme has been included, mainly for
testing during development of \ws. To assure numerical conservation of action,
a flux or control volume formulation is used. The flux between grid points
with counters $i$ and $i-1$ in $\phi$-space $(\cF_{i,-})$ is calculated as

% ------ First order scheme ------- %
% eq:1up_xy_1        Basic flux
% eq:1up_xy_2        Boundary velocity
% eq:1up_xy_3        Upstream value

\begin{equation}
\cF_{i,-} = \left [ \: \dot{\phi}_b \: \cN_u \: \right ]^n_{j,l,m}
\: , \label{eq:1up_xy_1} \end{equation} \begin{equation}
\dot{\phi}_b = 0.5 \: \left ( \dot{\phi}_{i-1} + \dot{\phi}_i 
\: \right )_{j,l,m} \: , \label{eq:1up_xy_2}
\end{equation} \begin{equation}
\cN_u = \left \{ \begin{array}{ccc}
\cN_{i-1} & \mbox{for} & \dot{\phi}_b \geq 0 \\
\cN_i     & \mbox{for} & \dot{\phi}_b   <  0
\end{array} \right . \: , \label{eq:1up_xy_3}
\end{equation}

\noindent
where $j$, $l$ and $m$ are discrete grid counters in $\lambda$-, $\theta$- and
$k$-spaces, respectively, and $n$ is a discrete time step
counter. $\dot{\phi}_b$ represents the propagation velocity at the `cell
boundary' between points $i$ and $i-1$, and the subscript $u$ denotes the
`upstream' grid point. At land-sea boundaries, $\dot{\phi}_b$ is replaced by
$\dot{\phi}$ at the sea point. Fluxes between points $i$ and $i+1$
$(\cF_{i,+})$ are obtained by replacing $i-1$ with $i$ and $i$ with
$i+1$. Fluxes in $\lambda$-space are calculated similarly, changing the
appropriate grid counters and increments.  The `action density' $(\cN^{n+1})$
at time $n+1$ is estimated as

% eq:1up_xy_tot

\begin{equation}
\cN_{i,j,l,m}^{n+1} = \cN_{i,j,l,m}^n
  + \frac{\Delta t}{\Delta \phi   } \left [ \cF_{i,-} - \cF_{i,+} \right ]
  + \frac{\Delta t}{\Delta \lambda} \left [ \cF_{j,-} - \cF_{j,+} \right ]
\: , \label{eq:1up_xy_tot}
\end{equation}

\noindent
where $\Delta t$ is the propagation time step, and $\Delta \phi$ and $\Delta
\lambda$ are the latitude and longitude increments, respectively. Equations
(\ref{eq:1up_xy_1}) through (\ref{eq:1up_xy_3}) with $\cN=0$ on land and
applying Eq. (\ref{eq:1up_xy_tot}) on sea points only automatically invokes
the required boundary conditions.

Note that Eq.~(\ref{eq:1up_xy_tot}) represents a two-dimensional
implementation of the scheme, for which the norm of the actual advection
vectors needs to be used in Eq.~(\ref{eq:dtpl}). Note furthermore, that this
implies a CFL criterion for the full equation, which is generally more
stringent than that for a scheme where $\lambda$ and $\phi$ propagation are
treated separately as in the third order schemes discussed below. For a grid
with equal increments in both directions, this results in a maximum time step
that is a factor $1/\sqrt{2}$ smaller for the first order scheme than for the
third order schemes.