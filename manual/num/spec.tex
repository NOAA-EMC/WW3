\vssub
\subsection{~Intra-spectral propagation} \label{sub:spec}
\vssub
\subsubsection{~General concepts}
\vsssub

The third step of the numerical fractional step algorithm considers refraction
and residual (current-induced) wavenumber shifts. Irrespective of the spatial
grid discretization and coordinate system, the equation to be solved in this
step becomes

%-----------------------------------%
% Step : Intra-spectral propagation %
%-----------------------------------%
% eq:step_intra
% eq:k_dot_g

\begin{equation}
\frac{\p N}{\p t} + \frac{\p}{\p k} \, \dot{k}_g N +
\frac{\p}{\p \theta} \, \dot{\theta}_g N = 0
\: , \label{eq:step_intra} \end{equation} \begin{equation}
\dot{k}_g  = \frac{\p \sigma}{\p d} 
    \frac{{\bf U} \cdot \nabla_x d}{c_g}  -
    {\bf k} \cdot \frac{\p {\bf U}}{\p s}
\: , \label{eq:k_dot_g} \end{equation}

\noindent
where $\dot{k}_g$ is the wavenumber velocity relative to the grid, and
$\dot{\theta}_g$ is given by (\ref{eq:theta_g_dot}) and (\ref{eq:theta_dot}).
This equation does not require boundary conditions in $\theta$-space, as the
model by definition uses the full (closed) directional space. In $k$-space,
however, boundary conditions are required. At low wavenumbers, it is assumed
that no wave action exists outside the discrete domain. It is therefore
assumed that no action enters the model at the discrete low-wavenumber
boundary. At the high-wavenumber boundary, transport across the discrete
boundary is calculated assuming a parametric spectral shape as given by
Eq. (\ref{eq:tail_N_k}). The derivatives of the depth as needed in the
evaluation of $\dot{\theta}$ are mostly determined using central
differences. For points next to land, however, one-sided differences using sea
points only are used.

Propagation in $\theta$-space can cause practical problems in an explicit
numerical scheme, as the refraction velocity can become extreme for long waves
in extremely shallow water or due to strong current shears. Similarly, 
the propagation in $k$-space suffers from similar problems in very shallow water. 
To avoid the need of extremely small time steps
due to refraction, the propagation velocities in $\theta$-space and $k$-space
(\ref{eq:theta_dot}) are filtered,

% eq:theta_filter

\begin{equation}
\dot{\theta} = X_{rd}(\lambda,\phi,k)\left( \dot{\theta_d} + 
\dot{\theta_c} + \dot{\theta_g} \right)\: , \label{eq:theta_filter} \end{equation}

\noindent
where the indices $d$, $c$ and $g$ refer to the depth, current and great-circle
related fraction of the refraction velocity in (\ref{eq:theta_dot}). The
filter factor $X_{rd}$ is calculated for every wavenumber and location
separately, and is determined so that the \cfl\ number for propagation in
$\theta$-space due to the {\em depth} refraction term cannot exceed a pre-set
(user defined) value (default 0.7). This corresponds to a reduction of the
bottom slope for some low frequency wave components. For mid-latitudes, the
affected components are expected to carry little energy because they are in
extremely shallow water. Long wave components carrying significant energy are
usually traveling toward the coast, where their energy is dissipated
anyway. This filtering is also important for short waves, and close to the
pole. The effect of this filter can be tested by reducing the time steps for
intraspectral refraction and by looking at the maximum CFL numbers in the
output of the model.  These are computed just before the filter is applied.

% \vspace{\baselineskip}
% \centerline{\ldots ADD DYNAMIC TIME INTEGRATION \ldots}


The spectral space is always discretized with constants directional increments
and a logarithmic frequency grid (\ref{eq:sigma_grid}) to accommodate
computations of the nonlinear interaction $S_{nl}$. First, second and third
orders schemes are available, and are presented in the following sections.