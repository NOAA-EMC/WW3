\vssub
\subsection{~Spatial propagation} \label{sub:xy_prop}
\vssub
\subsubsection{~General concepts}
\vsssub

Spatial propagation in \ws\ is described by the first terms of
Eq. (\ref{eq:bal_f_grid}). For spherical coordinates
[Eq. (\ref{eq:bal_sphere})], the corresponding spatial propagation step
becomes

%----------------------------%
% Step : Spatial propagation %
%----------------------------%
% eq:step_xy_prop

\begin{equation}
\frac{\p \cN}{\p t} + \frac{\p}{\p \phi} \, \dot{\phi} \cN +
\frac{\p}{\p \lambda} \, \dot{\lambda} \cN = 0
\: , \label{eq:step_xy_prop} 
\end{equation}

\noindent 
where the propagated quantity $\cN$ is defined as $\cN \equiv N \, c_g^{-1} \,
\cos\phi$. For the Cartesian grid, a similar equation is found for
$\cN \equiv N \, c_g^{-1}$. In this section equations for the more complicated
spherical grid are presented only. Conversion to a Cartesian grid is generally
a simplification and is trivial.

Equation~(\ref{eq:step_xy_prop}) in form is identical to the conventional
deep-water propagation equation, but includes effects of both limited depths
and currents. At the land-sea boundaries, wave action propagating toward the
land is assumed to be absorbed without reflection, and waves propagating away
from the coast are assumed to have no energy at the coastline. For so-called
`active boundary points' where boundary conditions are prescribed, a similar
approach is used. Action traveling toward such points is absorbed, whereas
action at the boundary points is used to estimate action fluxes for components
traveling into the model.

The spatial grids can use two different coordinate systems, either a `flat'
Cartesian coordinate system typically used for small scale and idealized test
applications, and a spherical (latitude-longitude) system used for most
real-world applications. In model version 3.14, the coordinate system was
selected at compile time with the {\code XYG} or {\code LLG} switches. In more
recent model versions, the grid type is now a variable defined in {\file
ww3\_grid} and stored in the {\file mod\_def.ww3} file. 

There is an option for spherical grids to have simple closure, to be periodic
 in the longitude direction, e.g. so that energy can propagate east from the maximum longitude in the grid to the minimum longitude in the grid. This closure is ``simple'' insofar as the index for latitude does not change across this ``seam''. A ``not simple'' type of closure is also permitted: this is associated with tripole grids. The tripole grid is a type of irregular grid and so this closure is discussed further in (\ref{sub:num_space_curv}).
 
Up to model version 3.14, \ws\ considered only regular discrete grids,
where the two main grid axes ($x,y$) are discretized using constant
increments $\Delta x$ and $\Delta y$. In model version \WWver\
additional options have been included, including curvilinear grids and
unstructured grids. In the following sections these grid approaches
will be discussed, before additional propagation issues are addressed,
covering the Garden Sprinkler Effect (\ref{sub:num_GSE}), continuously
moving grids (\ref{sub:num_move}) unresolved islands
(\ref{sub:num_obst}), and rotated grids (\ref{sub:num_space_rotagrid}).
