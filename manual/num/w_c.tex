\vssub
\subsection{~Winds and currents} \label{sub:num_w_c}
\vssub

\noindent
Model input mainly consists of wind and current fields. Within the model,
winds and currents are updated at every time step $\Delta t_g$ and represent
values at the end of the time step considered. Several interpolation methods
are available (selected during compilation). By default, the interpolation in
time consists of a linear interpolation of the velocity and the direction
(turning the wind or current over the smallest angle). The wind speed or
current velocity can optionally be corrected to (approximately) conserve the
energy instead of the wind velocity. The corresponding correction factor $X_u$
is calculated as

% eq:X_u10

\begin{equation}
X_u = \max \left [ \: 1.25 \: , \: \frac{u_{10,rms}}{u_{10,l}}
\right ] \: , \label{eq:X_u10} \end{equation}

\noindent
where $u_{10,l}$ is the linearly interpolated velocity and $u_{10,rms}$ is the
rms interpolated velocity. Finally, winds can optionally be kept constant and
changed discontinuously (option not available for current).

Note that the auxiliary programs of \ws\ include a program to pre-process
input fields (see \para\ref{sec:ww3prep}). This program transfers gridded
fields to the grid of the wave model. For winds and currents this program
utilizes a bilinear interpolation of vector components. This interpolation can
be corrected to (approximately) conserve the velocity or the energy of the
wind or the current by utilizing a correction factor similar to
Eq.~(\ref{eq:X_u10}).
