\vssub
\subsection{~Auxiliary programs} \label{sec:auxprog}
\vsssub
\subsubsection{General concepts}
\vsssub

All auxiliary programs presented here, with the exception of the track output
post-processor, read input from a pre-defined input file. The first character
on the first line of the input file will be considered to be the comment
character, identifying comment lines in the input file. This comment character
has to appear on the first position of input lines to be effective. In all
examples in the following sections lines starting with '{\tt \$}' therefore
only contain comment. The programs furthermore all write formatted output to
the standard output unit.

In the following sections, all available auxiliary programs are described
using an example input file with all options included (partially as
comment). These files are identical to the distributed example input
files. The sections furthermore show the name of the executable program, the
program name (as appears in the program statement), the source code file and
input and output files and their unit numbers (in brackets behind the file
name). Input and output files marked with \opt are optional. The intermediate
files mentioned below are all {\F unformatted}, and are not described in
detail here. Each file is written and read by a single routine, to which
reference is made for additional documentation.

\begin{list}{}{\itemsep 0mm \parsep 0mm \leftmargin 40mm \labelwidth 30mm}
\item[{mod\_def.ww3} \hfill] Subroutine {\F w3iogr} ({\file w3iogrmd.ftn}).
\item[{out\_grd.ww3} \hfill] Subroutine {\F w3iogo} ({\file w3iogomd.ftn}).
\item[{out\_pnt.ww3} \hfill] Subroutine {\F w3iopo} ({\file w3iopomd.ftn}).
\item[{track\_o.ww3} \hfill] Subroutine {\F w3iotr} ({\file w3iotrmd.ftn}).
\item[{restart.ww3}  \hfill] Subroutine {\F w3iors} ({\file w3iorsmd.ftn}).
\item[{nest.ww3}     \hfill] Subroutine {\F w3iobc} ({\file w3iobcmd.ftn}).
\item[{partition.ww3}\hfill] Subroutine {\F w3iosf} ({\file w3iosfmd.ftn}).
\end{list}

\noindent
Preprocessing and compilation of the programs is discussed in the following
two chapters. Examples of test runs of the model are provided with the source
code.

\pb
