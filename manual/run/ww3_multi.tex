\vsssub
\subsubsection{The multi-grid shell} \label{sec:ww3multi}
\vsssub

\proddefH{ww3\_multi}{w3mlti}{ww3\_multi.ftn}
\proddeff{Input}{ww3\_multi.nml}{Input file for multi-grid wave model: alternative namelist form.}{8}
\proddeff{Input}{ww3\_multi.inp}{Input file for multi-grid wave model: traditional format.}{8}
\proddeff{Output}{standard out}{Formatted output of program.}{6}
\proddefa{log.mww3}{Output log of wave model driver.}{9}
\proddefa{test.mww3\opt}{Test output of wave model.}{auto}

\vspace{\baselineskip}
\noindent
This wave model program requires and produces a plethora of input and output
files consistent with those of {\file ww3\_shel} in \para\ref{sec:ww3shel},
where file extensions {\file .ww3} are replaced by an identifier for a
specific grid. Note that all files are opened by name, and that the unit
number assignment is dynamic and automatic.

In order to make all existing features available there is a new version of the input file that uses namelists. This 
is the version that will be supported in the future as it allows a more flexible addition of new features. 
{\bf Please note that the namelist form is not supported by GCC compilers before version 4.8.2.} 

\nmlfile{ww3_multi.tel}

\pb

\inpfile{ww3_multi.tex}

\pb
