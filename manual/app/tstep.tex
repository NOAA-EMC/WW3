\pagestyle{myheadings} \setcounter{page}{1} \setcounter{footnote}{0}

\section{~Setting model time steps} \label{app:tstep}
\newcounters 
\vssub

Model time steps are set on a grid-by-grid basis and are considered as a part
of the model setup in the model definition file {\file mod\_def.ww3}. This
implies that in a multi-grid model set-up (using the model driver {\file
ww3\_multi}) each grid is associated with its own time step setting. In this
section some guidance is given for setting time steps for individual grids,
and for grids in a mosaic approach.  Examples of practical time step setting
for practical grids can be found in the individual grids used in the test
cases {\file mww3\_case\_01} through {\file mww3\_case\_03}.

\subsection{~Individual grids}

A basic wave model grid requires the definition of four time steps as is
described in \para\ref{sec:basic_num} on page \pageref{dt_list} of this
manual. Typically, the first step to consider is the \cfl\ time step for
spatial propagation, that is, the second of the four time steps defined in
{\file ww3\_grid.inp} for the grid considered. The critical \cfl\ number $C_c$
that identifies stability of the numerical scheme is defined as [compare
Eq.~(\ref{eq:quick_4})]

\begin{equation}
C_c = \frac{c_{g,\max} \Delta t}{\min(\Delta x , \Delta y)} \:\:\: , 
\label{eq:CFLmax}
\end{equation}

\noindent 
where $c_{g,\max}$ is the maximum group velocity, and $\Delta t$, $\Delta x$,
and $\Delta y$ are time and space increments. The maximum group velocity is
the group velocity for the lowest discrete model frequency. Noting that for a
given frequency the largest group velocity occurs in intermediate water depth,
this maximum velocity is approximately 1.15 times the deep water group
velocity for the lowest discrete spectral frequency. Note that the \cfl\
number formally includes affects of currents [Eq.~(\ref{eq:x_dot})] and grid
movement [Eq.~(\ref{eq:bal_move})]. The latter two effects are accounted for
internally in the model by adjusting the corresponding minimum time step
dynamically depending on the current velocity and the grid movement speed.
Hence, the user can define this minimum propagation time step ignoring
currents and grid movement. For the schemes used here the critical \cfl\
number is 1.

The second time step to consider is the overall time step (the first time step
identified in {\file ww3\_grid.inp}).For maximum numerical accuracy, this time
step should be set smaller than or equal to the above \cfl\ time step.
However, particularly in spherical grids, the critical \cfl\ condition occurs
only in a few grid points. In most grid points, \cfl\ numbers will be much
smaller. In such grids, accuracy does not suffer significantly if the overall
time step is take as 2 to 4 times the critical \cfl\ time steps. Such a
setting generally has a major positive impact on model economy. The key to
numerical accuracy is the interpretation of the \cfl\ number. this number
represents the normalized distance over which information propagates in a
single time step. Inaccuracy occurs if information propagates over several
grid boxes before source terms are applied. With \cfl\ $\approx 1$ and the
overall time step four times the \cfl\ time step, information will propagate
over four grid boxes before source terms are applied. This may lead to model
inaccuracies. If, however, the maximum \cfl\ number is 1, but the average
\cfl\ number is only 0.25, as is the case even for the lowest frequency in
many spherical grids, information only propagates over one grid box in a
single overall time step, and no issues with accuracy develop.

An effective overall time step also considers requested time intervals at
which model forcing is available, and at which model output is requested. If
input and output time steps are multiple integer times the overall time step,
a balanced and consistent numerical integration scheme exists, although the
model does not require this. Most important in this consideration is
reproducibility of results. If input or output time steps are modified so that
they are no longer an integer multiple of the overall model time step, then
the actual discrete time stepping in the model will be modified by these input
and output time steps, and hence an impact on actual model results may be
expected. Such an impact may be notable, but is generally very minor.

The third time step to consider is the maximum refraction (and wavenumber
shift) time step. For maximum model economy, this time step should be set
equal to (or larger than) the overall time step. However, this will alternate
the order of spatial and refraction computations for consecutive model time
steps, which in cases of strong refraction may lead to a minor oscillation of
wave parameter with a period of $2 \Delta t$. Such oscillations can be avoided
altogether by setting the maximum refraction time step to half the overall
time step. Considering the minor cost of the refraction term in the model,
this generally has a negligible impact on model economy. The preferred
refraction time step is therefore half the overall model time step.

One note of caution is appropriate with setting this time step. To assure
numerical stability, the characteristic refraction velocities are filter as in
Eq.~(\ref{eq:theta_filter}). This filtering suppresses refraction in cases
with rapidly changing bottom topography. The impact of this filtering is
reduced when the refraction time step is reduced. It is therefore prudent to
test a model grid with much smaller intra-spectral model time steps to assess
the impact of this filtering.

The final time step to set is the minimum time step for the dynamical source
term integration in \para\ref{sub:source}. This is a safety valve to avoid
prohibitively small time steps in the source term integration. Depending on
the grid increment size this is typically set to 5 to 15s. Note that
increasing this time step does not necessarily improve model economy; a larger
minimum source term integration time step will increase the spectral noise in
the integration, which in turn may {\em reduce} the average source term
integration time step!


\subsection{~Mosaics of grids}

Considerations for time step settings for individual grids making up a mosaic
model using {\file ww3\_multi} are in principle identical to those for
individual grids as discussed in the previous section. Additional
considerations are:

\begin{itemize}

\item Overall time steps for individual grids do not need to `match' in any
way for the management algorithm for the mosaic approach to work properly.
However, if identically ranked grids share overall time steps, and if integer
ratios between time steps of grids with different ranks are employed, then it
will be much easier to follow and predict the working of the management
algorithm,

\item If two grids with identical rank overlap, then the required width of the
overlap area will be defined by the stencil width of the numerical scheme, and
the number of times this scheme is called for the longest wave component
(ratio of overall time step to maximum \cfl\ time step). Thus, model economy
for individual grids will improve with increased overall model time step, but
the required overlap of equally ranked grids will then increase, reducing the
economy of the mosaic approach.

\end{itemize}

\bpage \pagestyle{empty}
