\vsssub
\subsubsection{~$S_{bs}$: Bottom scattering} \label{sec:BS1}
\vsssub

\opthead{BS1}{CREST model}{ F. Ardhuin}

\noindent
Waves propagating over a sloping bottom are partially reflected.  In the limit
of small variation in water depth $\Delta d$ with respect to the mean water
depth $d$, the reflection coefficient is proportional to the bottom spectrum
\cite{art:Kre49} and leads to a redistribution of wave energy in direction.
This process may be formulated as a source term, which leads to accurate
reflection coefficients when considering the evolution of the spectrum over
scales larger than the bottom auto-correlation length, with reasonable
accuracy up to $\Delta d / d \simeq 0.6$ \citep{art:AM07}. The source term
reads,

\begin{equation}
\cS_{\mathrm{bs}}({\bk}) =
\frac{\pi}{2}\int_{0}^{2 \pi}
 \frac{k'^2 M^2({\bk},\textbf{k}')}{\sigma \sigma' \left(k' C'_g +
{\bk}  \cdot {\bf U}  \right)}
 F^B({\bk}-\textbf{k}')
\left[N(\textbf{k}')- N(\textbf{k})\right] {\mathrm d}
\theta',\label{Sbscat}
\end{equation}

\noindent
with the coupling coefficient

\begin{equation}
M({\bk},{\bk}^\prime)\simeq M_b({\bk},{\bk}^\prime) =\frac{g {\bk}
\cdot {\bk}^\prime}{\cosh(kd)\cosh(k'd)}
\end{equation}

\noindent 
where the effect of bottom-induced current and elevation changes are
neglected, as appropriate for low to moderate current velocity relative to the
intrinsic phase speed, i.e. $U/C < 0.3$. For larger Froude numbers, in
particular in near-blocking conditions, the present implementation is not
expected to be accurate.  In Eq.~(\ref{Sbscat}), $k$ and $k'$ are related by
the resonance condition, $\omega=\omega'$, i.e. $\sigma+{\bk} \cdot {\bf U}
=\sigma' + {\bk}^\prime \cdot {\bf U}$, where ${\bf U}$ is the phase advection
velocity \cite[see, e.g.,][]{tol:WISE07}.

The bottom spectrum $F^B({\bk})$ is specified in the file {\file
bottom\_spectrum.inp}. This spectrum may be determined from multi-beam
bathymetric data.  In the absence of detailed bathymetric data, the sand dune
spectrum may be parameterized based on the work of \cite{art:Hino68}.  Recent
observations generally confirm the earlier data on sand dune spectra
\citep{art:AM07}, with a non-dimensional constant spectrum for large $k$,
i.e. $F^B({\bk}) \sim k^{-4}$.

The bottom spectrum is double-sided for simplicity of calculation
and normalized such that the bottom variance (in square meters) is

\begin{equation}
<d^2> =\int_{-\infty}^{\infty} \int_{-\infty}^{\infty}
F^B(k_x,k_y)  {\mathrm d} k_x  {\mathrm d} k_y.
\end{equation}

\noindent
In the present implementation this bottom spectrum is assumed to be the same
at all grid points.
 
The source term is computed according to different methods depending on the
value of the current. For zero current, the interactions only involves waves
of the same frequency and the interaction is always the same and linear in
terms of the directional spectrum. In this case the interaction is expressed
as a matrix problem. Namely, the directional spectrum $F$ is a vector of {\F NTH} components, 
and the source term is a vector of same size that is obtained by 
the matrix multiplication $S = M F$ with $M$ a square ({\F NTH,NTH}) array. This array is 
a function of the bottom spectrum and the non-dimensional depth $kd$.  This 
scattering matrix is precomputed and diagonalized as a
preprocessing step for a finite number of wavenumber magnitudes
\citep{art:AH02}.  The cost of this preprocessing increases linearly with the
number of discrete wavenumbers.

For non-zero current, the interaction pattern depends on the current magnitude
and direction (magnitude only for an isotropic bottom spectrum), and a precomputation of 
the scattering matrix would increase the overhead cost by at least one order of magnitude. In the
present implementation, the interaction integration with non-zero current is recomputed at every
source term call.
