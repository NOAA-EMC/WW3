\vsssub
\subsubsection{~$S_{ice}$: Empirical/parametric damping by sea ice} \label{sec:ICE4}
\vsssub

\opthead{IC4}{\ws/NRL}{C. Collins and E. Rogers}

\noindent
The fourth option ({\code IC4}) for damping of waves by sea ice was introduced by \cite{rep:CR17}. It gives methods to implement one of several simple, empirical/parametric forms for the dissipation of wave energy by sea ice.  The motivation for {\code IC4} is to provide a simple, flexible, and efficient source term which reproduces, albeit in a highly parameterized way, some basic physics of wave-ice interaction. The method is set by the integer value (presently 1 to 7) for {\code IC4METHOD} namelist parameter: 1) an exponential fit to the field data of \cite{art:WAD88}, 2) the polynomial fit in \cite{art:MBK14}, 3) a quadratic fit to the calculations of \cite{art:KM08} given in \cite{art:HT15}, 4) Eq. 1 of \cite{art:Ko14}, 5) a simple step function with up to 4 steps (may be nonstationary and non-uniform), and 6) a simple step function with up to 10 steps (must be stationary and uniform), and 7) a formula from \cite{art:Dob15} which uses ice thickness. All but the fourth method of {\code IC4} feature frequency-dependent attenuation. With the fourth method, attenuation varies with waveheight but is uniform in frequency space. 

In the following discussion we use {\code IC4M1} to denote {\code IC4} method 1, and so forth. {\code IC4} appears in the {\file switch} and namelist {\code IC4METHOD=1} (for example) appears in the file {\file ww3\_grid.inp}. Whereas in {\code IC1}, ${C_{ice,1}}$ is the user-determined attenuation, for {\code IC4M1}, {\code IC4M2}, and {\code IC4M4} ${C_{ice,n}}$ are constants of the equations. For {\code IC4M3}, ${C_{ice,1}}$ is ice thickness. For {\code IC4M5}, ${C_{ice,n}}$ controls the step function. Note that ${C_{ice,n}}$ may be provided by the user as non-stationary and non-uniform using methods analogous to methods used to input water levels.

{\code IC4M1}: an exponential equation was chosen to fit the data contained in table 2 of \cite{art:WAD88} which results in preferential attenuation of high frequency waves. This parameterizes the well-known low-pass filtering effect of ice. The equation has the following form:
\begin{equation}\label{eq:ice1}
  {\alpha} = \exp\left[\frac{-{2\pi}C_{ice, 1}}{{\sigma}} - C_{ice, 2}\right]
\end{equation}

\noindent Here, $\alpha$ is the exponential decay rate for energy, which is twice that for amplitude: $\alpha = 2k_i$. The values determined from the data are ${C_{ice,1...2}}=[0.18, 7.3]$, but these may be modified by the user. This method is described and applied in \cite{rep:CR17}.

{\code IC4M2}: In this method, the dissipation is represented using a user-specified polynomial. It is a powerful method, since many shapes can be represented, e.g. by fitting to observation-based dissipation rates. The method is described and applied in \cite{rep:CR17}. The equation is the following:
\begin{equation}\label{eq:ice2}
  {\alpha} = C_{ice,1} + C_{ice,2}\left[{\frac{\sigma}{2\pi}}\right] + C_{ice,3}\left[{\frac{\sigma}{2\pi}}\right]^2 + C_{ice,4}\left[{\frac{\sigma}{2\pi}}\right]^3 + C_{ice,5}\left[{\frac{\sigma}{2\pi}}\right]^4
\end{equation}

\noindent
If a user wishes to follow \cite{art:MBK14}, the suggested values for the coefficients are ${C_{ice,1...5}}=[0, 0, 2.12\times 10^{-3}, 0, 4.59\times 10^{-2}]$. Additional suggested polynomials can be found in \cite{rep:RMK18}.

With appropriate coefficients, this polynomial method can be used to reproduce the so-called “roll-over effect” where the attenuation is non-monotonic in frequency space. However, some recent studies do not indicate this effect, e.g. \cite{art:RTS16} and \cite{art:LK17}, and it may just be a spurious artifact in prior observational studies.

{\code IC4M3}: \cite{art:HT15} fit a quadratic equation to the attenuation coefficient calculated by \cite{art:KM08} as a function of frequency, $T$, and ice thickness, $h$. Attenuation increases for thicker ice and higher frequencies (lower periods). The number of  coefficients of the quadratic equation were prohibitively large to be user-determined, so the equation is hardwired in and the tunable parameter, ${C_{ice,1}}$, is ice thickness $h$. This method is described and applied in \cite{rep:CR17}. For reference, the equation is the following:
\begin{equation}\label{eq:ice3}
  {\ln{\alpha(T,h)}} = -0.3203 + 2.058h - 0.9375T - 0.4269h^2 + 0.1566hT + 0.0006T^2
\end{equation}

\noindent
There are two warnings to make about {\code IC4M3}. First, the equation itself was an extrapolation of the original range of $h$ used to calculate the attenuation coefficients in \cite{art:KM08} which was between 0.5 and 3 m, see \cite{art:HT15}. Second, in \cite{art:KM08}, wave attenuation predicted is based on scattering (a conservative process), whereas in {\code IC4M3}, the wave attenuation is treated as dissipation (non-conservative). This is ad hoc and not recommended for general use. Most especially, users should think twice before using {\code IC4M3} in combination with scattering routines {\code IS1} or {\code IS2}, since this is essentially double-counting scattering.

{\code IC4M4}: \cite{art:Ko14} found that attenuation was a function of significant wave height. Attenuation increased linearly with ${H_s}$ until ${H_s} = 3$ m at which point attenuation is capped, thus:
\begin{equation}
\left \{
\begin{array}{llrcl}
{\frac{\partial H_s}{\partial dx}} = {C_{ice,1}}\times {H_s}   & & \text{for} \> {H_s} \leq 3 \text{ m}  \\
{\frac{\partial H_s}{\partial dx}} = {C_{ice,2}}               & & \text{for} \> {H_s} > 3 \text{ m}     \\
\end{array} \right .
\end{equation}
where {$k_i=\frac{\partial H_s}{\partial dx}/H_s$}.

The values given in \cite{art:Ko14} are ${C_{ice,1...2}}=[5.35\times 10^{-6}, 16.05\times 10^{-6}]$. See regression test {\file ww3\_tic1.1/input\_IC4/M4} for examples. This method is described and applied in \cite{rep:CR17}.

{\code IC4M5}: This is a simple step function with up to 4 steps. It is controlled by the optionally nonstationary and non-uniform parameters ${C_{ice,1...7}}$. Parameters ${C_{ice,1...4}}$ control the step levels, which are in terms of dissipation rate, ${k_i}$. Parameters ${C_{ice,5...7}}$ control the step boundaries (given in Hz). See regression test {\file ww3\_tic1.1/input\_IC4/M5} for examples. This method is described in \cite{rep:CR17}.

{\code IC4M6}: This is a simple step function with up to 10 steps. It is controlled by the stationary and uniform namelist parameters {\code IC4KI} and {\code IC4FC}. Array {\code IC4KI} controls the step levels, which are in terms of dissipation rate, ${k_i}$, in radians per meter. Array {\code IC4FC} controls the step boundaries (given in Hz). See regression test {\file ww3\_tic1.1/input\_IC4/M6} for examples. 

{\code IC4M7}: This is a formula for dissipation from \cite{art:Dob15}, developed for a mixture of pancake and frazil ice, using data collected in the Weddell Sea (Antarctica). The formula depends on wave frequency and ice thickness:
\begin{equation}\label{eq:ice7}
  {\alpha=2k_i=0.2h^1f^{2.13}} \:\:\: .
\end{equation}
This method is described in \cite{rep:RPLA18}.

{\code IC4M8}: Like {\code IC4M7}, this method is in the general form of 
\begin{equation}\label{eq:ice8}
  {k_i=C_{hf}h^mf^n} \:\:\: .
\end{equation}
The formula is taken from \cite{Meylan2018}, where it is described as a ``Model with Order 3 Power Law''. It is applied by \cite{Liu2020}, where it is referred to as the ``M2'' model. The model specifies $m=1$ and $n=3$, and $C_{hf}$ is a user-specified calibration coefficient. \cite{Liu2020} provide calibration to two field cases and \cite{rep:RYW2021} provides a calibration to a third field case, \cite{art:RMK2021}. The third calibration is set as the default for {\code IC4M8}, $C_{hf}=0.059$, but can be changed in using the namelist parameter (constant and uniform) {\code IC4CN}, or using the spatially and/or temporally variable parameter  ${C_{ice,2}}$ . Further details on the calibrations are available in the inline documentation in {\file w3sic4md.F90}.  This method is functionally the same as the ``{\code M2}'' model in {\code IC5} (i.e., {\code IC5} with {\code IC5VEMOD=3}) and is redundantly included here as {\code IC4M8} because it is in the same ``family'' as {\code IC4M7} and {\code IC4M9}, being in the form of Eq. (\ref{eq:ice8}). 

For an example of setting the namelist parameter, see {\file /regtests/ww3\_tic1.1/input\_IC4\_M8}.

{\code IC4M9}: This formula is taken from the ``monomial power fit'' given in section 2.2.3 of \cite{rep:RYW2021}. Like {\code IC4M7} and {\code IC4M8}, it is a specific case of the general form of Eq. (\ref{eq:ice8}). The specificity is the constraint that $m=n/2-1$. This constraint is derived by \cite{rep:RYW2021} by invoking the scaling from \cite{art:YRW2019}, which is based on Reynolds number with ice thickness as the relevant length scale. This is also given as equation 2 in \cite{art:YRW2022}. The default namelist settings are $C_{hf}=2.9$ and $n=4.5$, from calibration by \cite{rep:RYW2021} to  \cite{art:RMK2021}. Further details, including alternative calibrations such as \cite{art:Yu2022}, are available in the inline documentation in {\file w3sic4md.F90}. Constant values can be set using namelist parameters, where $C_{hf}$ and $n$ are {\code IC4CN(1)} and {\code IC4CN(2)}, respectively. Spatially and/or temporally versions of the same can be specified as ${C_{ice,2}}$ and ${C_{ice,3}}$, respectively.

The namelist default $C_{hf}$ values in {\code IC4M8} and {\code IC4M9} are consistent with those of identical formulae implemented in \cite{man:SWAN4145A}.

  
