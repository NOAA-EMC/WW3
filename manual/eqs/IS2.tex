\vsssub
\subsubsection{~$S_{is}$: Scattering by sea ice} \label{sec:IS2}
\vsssub

\opthead{IS2}{\ws}{F. Ardhuin, P. Nicot, D. Dumont, C. Sevigne, T. Williams}

\noindent
The implementation of this scattering term generally 
follows the approach of \cite{art:MM06}. The first implementation here is based on \cite{art:KM08}, 
using 1D propagation, the rate of temporal attenuation in the initial direction is given by a 
parameter $\cS_{\mathrm{is}}({\bk}) = \beta_{s,\mathrm{MIZ}} N(\textbf{k})$, which is determined from pre-computed reflection coefficients 
over ice-water boundaries. The fraction of that energy which is re-distributed 
isotropically in all directions is defined by the  parameter $s_\mathrm{scat}$ wich is modified by 
{\code IS2BACKSCAT} in namelist SIS2. 

Additional scattering in the pack ice was introduced following \cite{art:SVB09} with a coefficient 
$\beta_{s,\mathrm{PACK}}=2 s_2 \exp(-s_3 / \sigma)$, where $s_2$ and $s_3$ are set by the 
{\code IS2C2} and {\code IS2C3} namelist parameters.  The total scattering coefficient is thus, 

\begin{equation}\label{eq:is2}
  \beta_{is} = \beta_{s,\mathrm{MIZ}} + \beta_{s,\mathrm{PACK}}
\end{equation}

In the end, the scattering-dissipation source term 
is a linear function of the spectrum. Namely, the directional spectrum $F$ is an {\F NTH} component vector, 
and the source term is a vector $S = M F$ with $M$ a square ({\F NTH,NTH}) array. Assuming an isotropic 
scattering, this linear scattering operator is easily diagonalized. There is one eigenvector that is 
the isotropic spectrum, for which the eigenvalue is 0, and the other eigenvectors have an eigenvalue equal 
to $s_\mathrm{scat} \beta$. It is thus straightforward to integrate both 
$\cS_{\mathrm{ice}}({\bk}) = \beta_{\mathrm{ice}}N(\textbf{k})$ and $\cS_{is}$ over any finite 
time step $\Delta t$. We have thus split the ice source terms from the other source terms, with a separate integration, 
\begin{equation}\label{eq:is2}
 N(t+\Delta t) = \exp(\beta_{\mathrm{ice}} \Delta t) \overline{N}(t)  + \exp \left[\left(\beta_{\mathrm{ice}} + \beta_{is} s_\mathrm{scat}\right) \Delta t\right] \left[N(t)-\overline{N}(t)\right]
\end{equation}
where $\overline{N}$ is the average over all directions. As a result, for a spatially homogeneous field, 
the spectrum exponentially tends to isotropy over a time scale $1/(\beta_{is} s_\mathrm{scat})$.



