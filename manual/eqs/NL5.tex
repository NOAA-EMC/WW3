\vsssub
\subsubsection{~$S_{nl}$: Generalized Kinetic Equation (GKE)} \label{sec:NL5}
\vsssub

\opthead{NL5}{WAVEWATCH III}{Q. Liu, O. Gramstad, A. Babanin}

\noindent
The Boltzmann integral formulation of \citet[][also known as the kinetic
equation; Eq. (\ref{eq:HKE}) - hereafter the HKE]{art:Has62} describes how
wave energy is redistributed over the spectral space due to \emph{resonant}
four-wave interactions (\ref{eq:resonance_2}). An important hypothesis
underpinning the derivation of the HKE is that the action density evolves on a
slow (``kinetic'') time scale of \textit{O}$(1 / \varepsilon^{4} \sigma_0)$,
where $\varepsilon$ and $\sigma_0$ are typical wave steepness and frequency of
the wave field \citep{Annenkov2006}. Consequently, as a large-time limit, the
HKE does not include contributions from \emph{non-resonant} interactions
($\Delta \sigma \neq 0$).

The HKE has been extended to include the \emph{non-resonant} four-wave
interactions by, for example, \citet{Janssen2003}, \citet{Annenkov2006} and
\citet[][hereafter GS13]{Gramstad2013}. The generalized kinetic equation (GKE)
implemented here is based on the work by GS13, \citet{Gramstad2016} and
\citet[][hereafter LGB21]{Liu2021JFM}. Following GS13, the GKE reads
%
\begin{align}
\frac{\mathrm{d} C_1}{\mathrm{d} t} &= 4 \Re \bigintssss \bigg[ T^2_{1, 2, 3, 4} \delta (\boldsymbol{\Delta k}) \/\mathrm{e}^{i\Delta \sigma t} I(t) \/  \bigg] \mathrm{d} \boldsymbol{k_{2, 3, 4}}, \label{eq:GKE}\\
%
I(t) &= \int_{0}^{t} \mathcal{F}_{1, 2, 3, 4} (\tau) \mathrm{e}^{-i\Delta \sigma \tau } \mathrm{d} \tau, \label{eq:GTINT}\\
%
\mathcal{F}_{1, 2, 3, 4} &= C_3 C_4 (C_1 + C_2) - C_1 C_2 (C_3 + C_4). \label{eq:GPROD}
\end{align}
%
Here $C_1 = C(\boldsymbol{k_1}) = g N(k, \theta) / k$ is the action spectrum,
$T_{1, 2, 3, 4} = T(\boldsymbol{k_1}, \boldsymbol{k_2}, \boldsymbol{k_3}, \boldsymbol{k_4})$
is the interaction coefficient \citep{art:Kra94, art:Jan09},
$\Delta \boldsymbol{k} = \boldsymbol{k_1} + \boldsymbol{k_2} - \boldsymbol{k_3} - \boldsymbol{k_4}$
is the wavenumber mismatch and $\mathrm{d} \boldsymbol{k_{2,3,4}} = \mathrm{d} \boldsymbol{k_2} \mathrm{d} \boldsymbol{k_3} \mathrm{d} \boldsymbol{k_4}$.
Note that
\begin{equation}
S_{nl}(k, \theta) = \omega \frac{\mathrm{d} N}{\mathrm{d}t} \bigg\rvert_{nl} = \frac{\omega k}{g} \frac{\mathrm{d} C_1}{\mathrm{d} t}.
\label{eq:GSnl}
\end{equation}
From the equations above, we see that the GKE not only includes the non-resonant
interactions [the $\delta (\Delta \sigma)$ existing in the HKE is eliminated in
Eq.~(\ref{eq:GKE})] but also suggests the evolution of a wave field depends
on its previous history of evolution (i.e., $I(t)$ in Eq.~(\ref{eq:GTINT})).
The full details of the GKE algorithm are described in GS13 and LGB21.
The modified kinetic equation of \citet{Janssen2003} was
implemented in the GKE framework as well. Parameters available in the
namelist {\F SNL5} are summarized in Table~\ref{tab:gke}. The reader is
referred to the regression tests {\code ww3\_ts1/input\_nl5\_growth} and
{\code ww3\_ts1/input\_nl5\_gaussian} for how to use this nonlinear term.

% -------------------------------------------------------------------
\begin{table}
    \footnotesize
    \begin{center}
        \begin{tabular}{|l|p{8.cm}|l|}
        \hline \hline
        Namelist par. & Description & Default\\
        \hline
        NL5DPT        & water depth (in m)                                                               & 3000 \\
        \hline
        NL5OML        & quasi-resonant cut-off factor $\lambda_c$ (\ref{eq:QuasiResQ})                   & 0.10  \\
        \hline
        NL5KEV        & equation solved - {\code 0: GKE} (\ref{eq:GKE}); {\code 1:} \citet{Janssen2003}  & 0     \\
        \hline
        NL5PMX        & \parbox{8.cm}{number of periods for phase mixing $N_{pm}$ -\\ {\code \{0: without phase mixing; -1: $N_{pm} = 1 / b_T$\}}} & 100 \\
        \hline \hline
    \end{tabular}
    \end{center}
    %
    \caption{Parameters contained in the {\F SNL5} namelist.}
    \botline
    \label{tab:gke}
\end{table}
% -------------------------------------------------------------------

\paragraph{Filtration of quadruplets.} To reduce the computational expense
of the GKE, we only consider wavenumber configurations satisfying
%
\begin{equation}
\left \lbrace
\begin{array}{rcl}
\boldsymbol{k_1} + \boldsymbol{k_2} &=& \boldsymbol{k_3} + \boldsymbol{k_4}\\
|\Delta \sigma| & \leq &\lambda_c \min (\sigma_1,\ \sigma_2,\ \sigma_3,\ \sigma_4)
\end{array}
\right.
\label{eq:QuasiResQ}
\end{equation}
%
as valid quadruplets. Here $\lambda_c$ is a cut-off factor used to filter
out quartets very far from resonance which contribute little to spectral
evolution.
For a given quadruplet, we choose $\boldsymbol{k_1}$, $\boldsymbol{k_2}$
and $\boldsymbol{k_3}$ at the wavenumber grid points, and $\boldsymbol{k_4}$
is naturally determined by (\ref{eq:QuasiResQ}). Unless otherwise specified,
the action density $C(\boldsymbol{k_4})$ is obtained through bilinear
interpolation \citep{vanVledder2006}. When $\boldsymbol{k_4}$ falls beyond
the spectral grid, we assume
\begin{equation}
C(k_4) =
\begin{cases}
0 & \textnormal{ for } k_4 < k_{\min}\\
C(k_{\max}) \left( \frac{k_4}{k_{\max}} \right)^{n/2 - 2} & \text{ for } k_4 > k_{\max}\\
\end{cases},
\label{eq:GBound}
\end{equation}
where $k_{\min}$ and $k_{\max}$ are the minimum and maximum wavenumbers
within in the spectral grid, and $n=-5$ is the prescribed high-frequency
power law for frequency spectrum [i.e., $E(f) \propto f^{n}$].

\paragraph{Phase mixing.} The GKE is generally solved by assuming wave phases
are intially completely uncorrelated, i.e., a cold start with $I(t=0) = 0$ (GS13).
As wave field evolves, the nonlinear four-wave interactions give rise to
deviation from Gaussianity \citep{Janssen2003}, resulting in non-zero
fourth-order cumulant and hence non-zero $I(t)$. Some physical processes
like wave breaking \citep[e.g.,][]{bk:Bab11} may uncorrelate phases (i.e.,
phase re-mixing) at certain times. GS13 and \citet{Gramstad2016} suggested
such effect could be incorporated in the GKE algorithm by restarting the
GKE  (setting $t=0$) every $N_{pm}$ characteristic wave periods
[in terms of $T_{m0, -1}$; Eq.~(\ref{eq:Tm0m1})] while keeping the action
spectrum unchanged. It is noteworthy that the phase re-mixing is also
desired to \emph{suppress the numerical instability of the GKE for long-term
simulations} (see LGB21). Two options for $N_{pm}$ have been made available:
$N_{pm}$ can either be a fixed constant [$N_{pm} \sim \textit{O} (10^2)$]
or explicitly depend on the dominant breaking probability $b_T$ by employing
$N_{pm} = 1 / b_T$ and $b_T$ is estimated with Eq.~(\ref{eq:bt}) by following
\citet{art:BBY01}.

\paragraph{Source term integration.} Third-generation spectral wave models
usually employ a semi-implicit or implicit integration scheme to calculate
the change of action density $\Delta N$ in the time step $\Delta t$ from
source terms, for which the diagonal term $D(k, \theta) = \partial \mathcal{S} (k, \theta) / \partial N(k, \theta)$
is required [Eq.~(\ref{eq:implicit_st}); see section~\ref{sub:source}].
Owing to the presence of the time integral $I(t)$ in (\ref{eq:GKE}) which
explicitly depends on the history of the evolution of wave spectrum
[in terms of the action product term $\mathcal{F}_{1,2,3,4} (\tau)$], it is
challenging, if not impossible, to compute the diagonal term $D(k, \theta)$
for the GKE. To circumvent this problem, we adopt the explicit dynamic
time-stepping scheme [i.e., $\epsilon = 0$ in Eq.~(\ref{eq:implicit_st})]
of \citet{tol:JPO92} for source term integration when $S_{nl}$ is based
on the GKE.

\paragraph{\textit{\underline{Limitations of the code:}}} the GKE is
extremely expensive and thusfar can only be applied to a single-grid-point
version of WW3 (e.g., the duration-limited test). Owing to i) the inclusion of
quasi-resonant interactions and ii) the usage of the explicit source term
integration scheme, the GKE runs are approximately 5-10 times more
expensive than the WRT (section~\ref{sec:NL2}).
% <TODO> - regtest
