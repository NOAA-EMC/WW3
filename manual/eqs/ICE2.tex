\vsssub
\subsubsection{~$S_{ice}$: Damping by sea ice (Liu et al.)} \label{sec:ICE2}
\vsssub

\opthead{IC2}{\ws/NRL}{E. Rogers, S. Zieger, F. Ardhuin}

\noindent
This method for representing the dissipation of wave energy by wave-ice interaction is based on the papers
by \cite{art:LMC88}, \cite{art:LHV91} and \cite{art:Aea15}. The main input ice parameters
is the ice thickness (in meters) that can vary spatially and temporally and is the forcing field  ${C_{ice,1}}$. 

This is a model for attenuation by a
sea ice cover, derived on the assumption that dissipation is caused by
friction in the boundary layer below the ice,
with the ice modeled as a continuous thin elastic plate. The original form by  \cite{art:LMC88} is activated by 
setting the {\code IC2} namelist {\F SIC2} parameter {\code IC2DISPER = .TRUE.}. That form 
 assumes that the boundary layer is always laminar but it uses an eddy viscosity ${\nu}$ that can vary spatially 
and is the forcing field ${C_{ice,2}}$. 

 {\code IC2} and {\code IS2} share 
the optional use of the \cite{art:LMC88} dispersion relation for unbroken ice, 
\begin{equation}
\sigma^2 =  \left(gk_{ice} + B k_{ice}^5\right)  / \left(1/\tanh ( k_{ice}H) +\frac{\rho_{ice}  h k_{ice}} {\rho_{w} }\right),
\end{equation}
\begin{equation}
c_g =  (g+(5 + 4 k_{ice} M)Bk_{ice}^4)/(2\sigma(1+k_{ice}M)^2).
\end{equation}

\noindent
$B$ and $M$ quantify the effects of, respectively, ice bending due to waves and ice inertia. 
The group velocity under the ice, derived from the same relation, is used in the module {\code W3SIS2MD} and computed in  
{\code W3DISPMD}.  See \citet{art:LMC88} for details.

For {\code IC2}, the dispersion relation is defined by 

\begin{equation}\label{eq:ice1}
  {\sigma}^2 = ({gk_r} + {Bk_r^5})/(\coth({k_r}{h_w}) + {k_r}{M}),
\end{equation}
\begin{equation}\label{eq:ice2}
  {c_g} = (g + (5 + 4{k_r}{M}){B}{k_r^4})/(2{\sigma}(1+{k_r}{M})^2),
\end{equation}
\begin{equation}\label{eq:ice3}
  {\alpha} = (\sqrt{{\nu\sigma}}{k_r)}/({c_g}\sqrt{2}(1+{k_r}{M})).
\end{equation}

\noindent
In our notation, $h_w$ is water depth and $h_i$ is ice thickness.  The
variables $B$ and $M$ quantify the effects of the bending of the ice and
inertia of the ice, respectively. Both of these variables depend on $h_i$ 
\citep[see][]{art:LMC88, art:LHV91}.

This equation is only solved when ICEDISP=TRUE in the {\F MISC} namelist. Otherwise, $k_{ice}=k$, just like in open water. 
Note that the effect of $k_{ice}$ is not passed back to the main program. 
% I temporarily removed text "...is limited to wave breaking and dissipation..." : is this true? Or did the author mean to say "ice breaking and dissipation"? 

 \cite{art:SAG16} improved {\code IC2} by adding a better alternative to the eddy viscosity representation of dissipation. 
They distinguish between laminar and 
turbulent regimes, allowing this is activated by setting  {\code IC2DISPER = .FALSE.}. 
In that case the dissipation goes from a laminar form using the molecular viscosity multiplied by an 
empirical adjustment factor {\code IC2VISC} to a turbulent form, amplified by a factor {\code IC2TURB}, for Reynolds numbers 
above a user-defined threshold {\code IC2REYNOLDS}. This transition is smoothed over a range {\code IC2SMOOTH} to take into 
account the random nature of the wave field. In the turbulent regime, the friction factor 
is estimated from a user-specified under-ice roughness length {\code IC2ROUGH}, expected to be of the order of $10^{-4}$~m. 
Possibility has also been added to perform a heuristic reduction of the dissipation 
rate when the floe diameters are much shorter than the ocean wave 
wavelength \citep[see][for details]{Aea18}. It relies on the hypothesis that, in these conditions, ice floes follow at least partly 
the horizontal motion and it is thus logical to reduce the relative velocity 
between the ice and the water from which the dissipation rate is estimated. 
This reduction is controlled by the factor called $r_D$ in \cite{Aea18} which value can be set through the namelist parameter {\code IC2DMAX}. 
Reduction of the dissipation rate occures for waves longer than $D_{\max}/r_D$ ($D_{\max}$ 
being the maximum floe size), and there is no reduction for maximum floe diameters of the order of 
the wavelength or more.

The parameter {\code IC2TURBS} is an ad hoc enhancement of turbulent dissipation in the Southern hemisphere 
that was introduced for test purposes to investigate sources of bias. This will be deprecated in future versions. It now appears 
that combining IC2 with ineleastic dissipation in IS2 can provide good results for dominant waves in both hemispheres \citep{Aea18}. 

A full description of the original form of {\code IC2} can be found in \cite{rep:RO13}. 
It is applied in \cite{art:LKS15}. A concise summary of the present code is given 
in \cite{rep:RPLA18}. A description of the improvement of the dissipation mechanism 
of {\code IC2} can be found in Appendix B of \cite{art:SAG16}, where it is used.
