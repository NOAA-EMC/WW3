\vssub
\subsection{~Running test cases} \label{sec:tests}
\vssub

If \ws\ is installed and compiled successfully, it can be tested by running
most different program elements interactively from the {\file work}
directory. The switch settings in the generic switch file correspond to the
activated inputs in the example input files. It should therefore be possible
to run all model elements by typing
\command{ww3\_grid | more \\
  ww3\_strt | more \\
  ww3\_bound | more \\
  ww3\_prep | more \\
  ww3\_shel | more \\
  ww3\_outf | more \\
  ww3\_outp | more \\
  ww3\_ounf | more \\
  ww3\_ounp | more \\
  ww3\_trck | more \\
  ww3\_grib | more \\
  gx\_outf | more \\
  gx\_outp | more } where the {\code more} command is added to allow for
on-screen inspection of the output. This {\code | more} can be replaced by
redirection to an output file, e.g. \command{ww3\_grid > ww3\_grid.out } Note
that {\code ww3\_grib} will only provide GRIB output if a user-supplied
packing routine is linked in. Note furthermore that no simple interactive test
case for {\file ww3\_multi} is provided. GrADS can then be run from the work
directory to generate graphical output for these calculations. All
intermediate output files are placed in the {\file work} directory, and can be
removed conveniently by typing \command{w3\_clean}

\vspace{\baselineskip}
\noindent
\ldots I removed the old test cases here. We now need a short description of
regtests and cases \ldots

